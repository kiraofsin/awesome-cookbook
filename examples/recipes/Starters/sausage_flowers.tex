\recipe[style=style3]{Sausage Flowers}

\info[servings=24,
		time = 35,
		source = mom]{}

\recipefigure[style = background]{sausageflower.jpg}

\begin{ingredientsh}
	\ingredient{340}{g}{pound ground Italian sausage}
	\ingredient{170}{g}{monterey Jack cheese}
	\ingredient{170}{g}{Colby cheese}
	\ingredient{250}{ml}{salsa}
	\ingredient{24}{}{wonton wrappers}
	\ingredient{255}{g}{sour cream}
	\ingredient{1}{}{green onions (bunch)}
\end{ingredientsh}


\begin{preparation}
	\step Preheat oven to 350 degrees F (175 degrees C). Lightly grease a miniature muffin pan.
	
	\step Place ground Italian sausage in a large, deep skillet. Cook over medium high heat until evenly brown. Drain and remove from heat.
	
	\step Stir Monterey Jack cheese and Colby cheese into the warm sausage to melt. Stir in salsa.
	
	\step Gently press wonton wrappers into the prepared miniature muffin pan so that the edges are extending. Place a heaping tablespoon of the sausage mixture into each wonton wrapper.
	
	\step Bake 10 minutes in the preheated oven, or until wonton edges begin to brown.
	
	\step Transfer baked filled wontons to a serving platter. Dollop each with approximately 1 tablespoon sour cream. Sprinkle with green onions.
\end{preparation}

\begin{notes}
	\note{This recipe uses \texttt{style3} header and the \texttt{ingredienth} environment.}
	\note{There is also a picture, using \texttt{recipefigure} with the \texttt{background} style. You must be carefull where you include the image in your text file. It may go over or under the text depending on it's position.}
	\note{Real source: \href{http://allrecipes.com/recipe/26945/sausage-flowers/}{link}}
\end{notes}
