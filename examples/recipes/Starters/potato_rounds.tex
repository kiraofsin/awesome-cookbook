\recipe[style=style1]{Potato Rounds}

\info[servings=4,
		time = 60, 
		energy = 806, 
		urlsource = http://allrecipes.com/recipe/18344/cheese-and-bacon-potato-rounds/]{}

\begin{ingredientsh}
	\ingredient{4}{}{potatoes (baking)}
	\ingredient{60}{g}{butter}
	\ingredient{8}{slices}{bacon (cooked)}
	\ingredient{220}{g}{chaddar cheese}
	\ingredient{120}{gr}{green onions}
\end{ingredientsh}

\begin{preparation}
	\step Preheat oven to 400 degrees F (200 degrees C).
	
	\step Cut the potatoes in thick slices (1-1.5cm).
	
	\step Brush both side of potato slices with butter; place them on an ungreased cookie sheet. Bake in the preheated 400 degrees F (200 degrees C) oven for 30 to 40 minutes or until lightly browned on both sides, turning once.
	
	\step When potatoes are ready, top with bacon (cut), cheese, and green onion (cut); continue baking until the cheese has melted.
\end{preparation}


\begin{alternatives}
	\alternative Instead of cooked bacon you may use raw and bake it before adding it in~\refstep{4}.
	\alternative Use your preferred cheese instead of chaddar.
\end{alternatives}

\begin{notes}
	\note{This recipe uses a \texttt{style1} header and \texttt{ingredientsh} environment. The recipe is from a website as you can see in the source link.}
	\note{Note you can refer to steps in the preparation and in the alternative, see alternative~\refaltstep{1}.}
	\note{None of these "sections" (icons, ingredients, preparation, alternatives, notes) are mandatory, you can leave everything out or add custom LaTeX code in a recipe.}
\end{notes}

